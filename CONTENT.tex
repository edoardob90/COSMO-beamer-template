
% Content begins...

\section{Première section}

\begin{frame}{Nom du conférencier}{À propos du conférencier suivant}

  \begin{itemize}
  \item
    Affiliation actuelle du conférencier

    % Exemples:
    \begin{itemize}
    \item
      Professeur de Mathématiques, Université de Quelque Part.
    \item
      Partenaire Junior de la Compagnie X
    \item
      Conférencier pour l'organisation/le projet X.
    \end{itemize}
  \item
    Expérience et réalisation
    % Facultatif. Utilisez ceci s'il est approprié de flatté légèrement le
    % conférencier ; par exemple, s'il s'agit d'un invité.
    % En utilisant des sous-item, la liste des choses faites par le conférencier
    % paraît intéressante et suffisante.

    % Exemples:
    \begin{itemize}
    \item
      Diplôme d'Académie, mais uniquement si c'est approprié
    \item
      Positions actuelles ou futures, éventuellement avec les dates
    \item
      Publications (éventuellement juste le nombre de publications)
    \item
      Récompenses, prix
    \end{itemize}
  \item
    Concernant le sujet d'aujourd'hui
    % Facultatif. Il est important pour l'exposé d'utiliser ce point en dehors des toutes
    % expériences/connaissances spécifiques du conférencier et de ne pas poursuivre ce qui
    % a été indiqué plus haut.

    % Exemples:
    \begin{itemize}
    \item
      Expert qui a travaillé sur le projet durant X mois/année.
    \item
      Va présenter les recherches de son groupe/sa compagnie sur le
      sujet.
    \item
      Va résumer le rapport du projet ou le statut actuel du projet.
    \end{itemize}
  \end{itemize}
\end{frame}